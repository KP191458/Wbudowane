\documentclass{classrep}
\usepackage[utf8]{inputenc}
\usepackage{color}
\usepackage{gensymb}

\studycycle{Informatyka, studia niestocjanarne, I st.}
\coursesemester{IV}

\coursename{Systemy Wbudowane}
\courseyear{2019/2020}

\courseteacher{dr inż. Michał Morawski}
\coursegroup{Niedziela, 11:45-13:15}

\author{
  \studentinfo{KONRAD PŁAWIK}{191458} \and
  \studentinfo{Vladislav Mazur}{199185} \and
  \studentinfo{Łukasz Połubiński}{211833}
}

\title{Podlewaczka}
\svnurl{https://github.com/KP191458/Wbudowane}

\begin{document}
\maketitle
\newpage
\tableofcontents
\newpage

\section {Opis projektu}
Zadaniem projektu bylo stworzenie urządzenia które służyć będzie jako pomoc w monitorowaniu temperatury i wilgotności w ogrodzie bądź szklarni i dozowaniu zasobów wodnych. Urządznie mierzyć będzie temperaturę otoczenia i wilgotność powietrza. Pomierzone dane są wywietlane na ekranie a także urządzenie posiada dwie diody które sygnalizują przekrocznie ustalonego zakresu wilgotnosci. Przekroczenie zakresu uruchania także serwomechanizm, który może służyć jako urządzenie otwierające zawór nawodnienia szklarni.\\

\subsection {Wykaz urządzeń}
\begin{itemize} 
  \item Arduino Uno
  \item SHT31 - cyfrowy czujnik temperatury i wilgotności
  \item Arduino-Dem - moduł wyświetlacza LCD 2''
  \item Serwomechanizm Tower Pro SG90
  \item Diody LED 5 mm
  \item Rezystory przewlekane 330 \ohm
  \item Płytka stykowa 830 otworów
  \item Przewody połączeniowe męsko-męskie
  \item Przewody połączeniowe damsko-męskie
\end{itemize}

\subsection {Wykaz funkcjonalności}
\begin{enumerate}
  \item SPI
  \item GPIO
  \item LCD
  \item SHT31 - temperatura
  \item SHT31 - wilgotność
  \item Serwomechanizm
  \item LED
\end{enumerate}

\section {Dokumentacja użytkownika}
\subsection {Wymagania sprzętowe}
Projekt wymagał xxxx co wymagało modelu Arduino który posiadał co najmniej taką ilość wejść i wyjść. Zdecydowalimy się na użycie Arduino Uno, ponieważ posiadało ono wymaganą liczbę wejść i wyjść oraz, ze względu na popularność posiadało liczne dokumentacje i przykłady użycia.

\subsection {Instrukcja użytkownika}
Po podłączeniu zasilania zainstalowane na urządzeniu oprogramowanie rozpoczyna inicjalizację urządzeń a następnie pomiar temperatury i wilgotności.

\begin{itemize}
  \item Pomiar wykonywany jest co 1 sekundę 
  \item Odczytane pomiary wywietlane są na wyświetlaczu LCD
  \item Dioda LED w kolorze zielonym informuje że zmieziona wilgotność jest poniżej 70\%.
  \item Dioda LED w kolorze czerwonym informuje że zmieziona wilgotność jest powyżej 70\%.
  \item Po przekroczeniu wilgotności 70\% uruchomiony zostaje serwomechanizm który wykorzystany może zostać do otwarcia zaworu systemu nawadniającego (w szklarni bądź pomieszczeniu)
\end{itemize}

\subsection {Interfejs użytkownika}
Urządzenie nie wymaga interakcji użytkownika w celu dokonania pomiarów. Interfejsem jest ekran oraz kolor diod LED.\\

Z ekranu możemy odczytać wartości liczbowe temperatury (w stopniach Celsjusza) oraz wilgotności (w procentach).

\section {Specyfikacja urządzeń}

\subsection {Arduino Uno}
Arduino Uno to płyta mikrokontrolera oparta na ATmega328. Posiada 14 cyfrowych wejść / wyjść (z których 6 można wykorzystać jako wyjścia PWM), 6 wejść analogowych, ceramiczny rezonator 16 MHz, złącze USB, gniazdo zasilania, nagłówek ICSP i przycisk resetowania.\\

Arduino Uno w wersji R3 to najnowsza wersja po Duemilanove, z ulepszonym układem interfejsu USB. Podobnie jak Duemilanove, ma on nie tylko rozszerzony nagłówek osłony z napięciem odniesienia 3,3 V i pin RESET (który rozwiązuje problem dotarcia do szpilki RESET w osłonie) oraz bezpiecznik 500 mA do ochrony portu USB komputera, ale także automatyczny obwód do wyboru zasilania USB lub DC bez zworki! Uno jest kompatybilny z kodami pin i Duemilanove, Diecimilla i starszymi Arduino, więc wszystkie płytki (shieldy), biblioteki i kod będą nadal działać. Nowy R3 (trzecia wersja) UNO ma kilka drobnych aktualizacji, z aktualizacją układu interfejsu USB i dodatkowymi przełamaniami dla pinów i2c i pinów IORef.\\

Arduino to platforma prototypowania elektroniki typu open source oparta na elastycznym, łatwym w użyciu sprzęcie i oprogramowaniu. Jest przeznaczony dla artystów, projektantów, hobbystów i wszystkich zainteresowanych tworzeniem interaktywnych obiektów lub środowisk.\\

Arduino może wyczuwać środowisko, otrzymując dane wejściowe z różnych czujników i może wpływać na otoczenie, kontrolując światła, silniki i inne siłowniki. Mikrokontroler na płycie programowany jest za pomocą języka programowania Arduino (opartego na Wiring) i środowiska programistycznego Arduino (opartego na Processing). Projekty Arduino mogą być autonomiczne lub mogą komunikować się z oprogramowaniem działającym na komputerze (np. Flash, Processing, Max / MSP).\\

Parametry techniczne:
\begin{itemize}
  \item Mikrokontroler ATmega328
  \item Napięcie robocze 5 V.
  \item Napięcie wejściowe (zalecane) 7-12 V.
  \item Napięcie wejściowe (limity) 6-20 V.
  \item Cyfrowe piny we / wy 14 (z których 6 zapewnia wyjście PWM)
  \item Piny wejścia analogowego 6
  \item Prąd DC na pin we / wy 40 mA
  \item Prąd DC dla 3,3 V Pin 50 mA
  \item Pamięć flash 32 KB (ATmega328) z czego 0,5 KB jest używane przez bootloader
  \item SRAM 2 KB (ATmega328)
  \item EEPROM 1 KB (ATmega328)
  \item Szybkość zegara 16 MHz
  \item Długość 68,6 mm
  \item Szerokość 53,4 mm
  \item Waga 25 g
\end{itemize}

\subsection {SHT31 - cyfrowy czujnik temperatury i wilgotności}
Seria czujników temperatury i wilgotności SHT3x dostępna jest w dwóch wersjach: wersja niskokosztowa SHT30 i wersja standardowa sht31. Rodzina SHT3x łączy wiele funkcji i interfejsów z przyjaznym szerokim zakresem napięć roboczych (2.4 do 5.5 V). W porównaniu do czujników poprzedniej generacji seria SHT3x jest inteligentniejsza, bardziej niezawodna i dokładniejsza. Jest bardziej funkcjonalny, ma wyższe możliwości przetwarzania sygnału i może odczytywać wartości temperatury i wilgotności za pomocą różnych pinów.\\

Ponadto wprowadzenie tej wersji dodatkowo rozszerzyło rodzinę produktów SHT3x. Zastosowanie osłony ochronnej typu filmowego rozszerza również zakres zastosowań. Jest to stopień ochrony IP67 dla wody i ochrona przeciwpyłowa niż folia PTFE, dzięki czemu może być stosowany w trudnych warunkach, w których woda i kurz mogą wpływać na dokładność i wydajność czujnika.\\

Cechy:
\begin{enumerate}
  \item Wysoka niezawodność i długa stabilność
  \item Z technicznego punktu widzenia niezawodne
  \item Nadaje się do aplikacji wysoka głośność
  \item Wysoka zdolność przetwarzania sygnału
  \item Niski poziom sygnału\\
\end{enumerate}


Parametry produktu:
\begin{enumerate}
  \item Wyjście: I2C, napięcie wyjściowe
  \item Napięcie zasilania: 2.4 do 5.5 V
  \item Zakres roboczy RH: 0-99.99\% RH
  \item Zakres temperatury pracy:-40 \degree do 125 \degree c (-40 \degree do 257 \degree f)
  \item Czas reakcji RH: 8 sekund (tau63 \%)
\end{enumerate}


\subsection {Arduino-Dem - moduł wyświetlacza LCD 2''}
DEM 1280640 FGH-PW\\

FUNKCJE ORAZ CECHY\\
LCD TYPE
Transflective Positive Mode\\
Viewing Direction           : 6 o'clock\\
 Driving Scheme           : 1/65 Duty, 1/9 Bias\\
 Power Supply Voltage   : 3.3 Volt\\
 VLCD                        : 9.0 Volt\\
 Display Contents         : 128 x 64 Dots\\
 Driver IC                   : ST7565R (Sitronix)\\
 RoHS                        : Compliant\\
 Interfejs                    : SPI\\

SPECYFIKACJE MECHANICZNE\\
Module Size   	:  58.20 x 41.70 x 5.70 mm\\
Viewing Area 	:  50.00 x 25.00 mm\\
Active Area 	:  46.06 x 23.02 mm\\
Dot Size		: 20.34. x 0.34 mm\\
Dot Gap 		: 20.02 mm\\

\subsection {Serwomechanizm Tower Pro SG90}
9-gramowe serwo typu micro SG90 firmy TowerPro to bardzo popularny model serwa stosowany w modelarstwie, robotyce, konstrukcjach napędowych, systemach alarmowych.\\

Serwo ma niewielką wagę lecz jest stosunkowo szybkie i ma siłę odpowiednią do sterowania modelami.\\

Charakterystyka:
\begin{itemize}
  \item Prędkość przekładni: 0,12 s/60\degree (4,8 V)
  \item Moment: 1,2 – 1,8kg/cm (4,8 V)
  \item Napięcie pracy: 4,8 – 7,2V
  \item Długość przewodów zasilania: 23,5cm
  \item Waga: 9g
  \item Czas reakcji: 7 us
  \item Temperatura pracy: -30 do +60 stopni
  \item Wymiary 22mm x 12mm x 22,7mm\\
\end{itemize}

Oznaczenie przewodów:
\begin{itemize}
  \item Czerwony: zasilanie
  \item Brązowy: masa
  \item Pomarańczowy: sygnał
\end{itemize}

\subsection {Diody LED 5 mm}
Parametry diody zielonej:
\begin{itemize}
  \item Soczewka w kolorze zielonym
  \item Obudowa: DIP 5 mm
  \item Długość emitowanej fali: 571 nm
  \item Jasność: 100 - 150 mcd
  \item Kąt świecenia: 50\degree
  \item Temp. pracy: od-40 \degree C do +80 \degree C
  \item Parametry pracy:
\begin{itemize}
  \item Prąd If: 20 mA
  \item Napięcie Vf: 2,3 - 2,5 V\\
\end{itemize}
\end{itemize}

Parametry diody czerwonej:
\begin{itemize}
  \item Soczewka w kolorze czerwonym
  \item Obudowa: DIP 5 mm
  \item Długość emitowanej fali: 625-645 nm
  \item Jasność: 450 - 800 mcd
  \item Kąt świecenia : 70 \degree
  \item Temp. pracy: od -40 \degree C do +80 \degree C
  \item Parametry pracy:
\begin{itemize}
  \item Prąd If: 20mA
  \item Napięcie Vf: 2,0 - 2,3 V'
\end{itemize}
\end{itemize}

\subsection {Rezystor przewlekany 330 \ohm}
Specyfikacja rezystorów
\begin{itemize}
  \item Rezystancja: 330 \ohm
  \item Moc znamionowa: 1/4 W
  \item Tolerancja: 5\%
  \item Montaż: przewlekany THT
\end{itemize}
\subsection {Płytka stykowa 830 otworów}
Dane techniczne płytki stykowej
\begin{itemize}
  \item Wymiary: 165 x 53 mm
  \item Liczba otworów: 830
  \item Posiada kolorowe paski, które mogą oznaczać polaryzację zasilania (+i-)
\end{itemize}

\begin{thebibliography}{0}
\end{thebibliography}
{\color{blue} 
Na końcu należy obowiązkowo podać cytowaną w sprawozdaniu
literaturę, z której grupa korzystała w trakcie prac nad zadaniem (przykład na
końcu szablonu)}
\end{document}
